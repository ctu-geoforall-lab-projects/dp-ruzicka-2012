\chapter*{Závěr}
Výsledkem této diplomové práce je nástroj Workflow Builder pro QGIS
Processing Framework. Nástroj dává uživateli možnost grafickou cestou
spojovat již existující moduly a vytvářet tímto způsobem moduly
nové. Workflow Builder pracuje s moduly, které jsou dostupné přes
rozhraní QGIS Processing Framework. Workflow Builder umožňuje
uživateli takto sestavené workflow uložit pro další použití. Při
spojování modulů je na první pohled vidět, který parametr je s kterým
spojen. Výsledný modul lze znova otevřít a upravovat jej.

Vývoj rozhraní QGIS Processing Framework se ale bohužel zastavil s
uvolněním implementace knihovny SEXTANTE pro Quantum GIS. Knihovna
SEXTANTE je psaná v jazyce Java a pro Quantum GIS byla přepsána do
Pythonu. Verze {\index{SEXTANTE}SEXTANTE} pro Quantum GIS se objevila
na konci psaní této práce. Daný framework má podobné cíle. Jeho velká
výhoda oproti QGIS Processing Framework je v tom, že SEXTANTE existuje
již několik let a zahrnuje podporu pro funkce z dalších GIS nástrojů
jako SAGA či GRASS a má v sobě implementovány také další knihovny s
geoalgoritmy (GDAL, fTools).

Během práce na Workflow Builderu pro QGIS Processing Framework jsem se
více seznámil s knihovnou Qt, jejím Graphics View Frameworkem a
architekturou MVC.

V současné době chybí možnost seskupování modulů ve scéně, to by mohlo
při větších workflow zpřehlednit scénu. Pakliže by se práci na QGIS
Processing Frameworku pokračovalo, rád bych tyto možnosti doplnil.

Ilustrační video pro práci s \textbf{Workflow Builder}em zde:

\begin{center}
	\href{http://youtu.be/4PxvWvTIyaU}{\texttt{http://youtu.be/4PxvWvTIyaU}}
\end{center}
 
\newpage
Aktuální verzi Workflow Builderu můžeme najít na přiloženém cd, případně zde:

\begin{center}
	\href{https://github.com/CzendaZdenda/qgis}{\texttt{https://github.com/CzendaZdenda/qgis}}
\end{center}
