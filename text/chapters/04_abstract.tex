\section*{Abstrakt}\label{abstrakt}
\indent Diplomová práce si vytyčila za cíl vytvořit v prostředí
Quantum GIS (dále jen QGIS) nástroj, který by umožňoval uživateli
grafické propojování modulů z frameworku \textbf{QGIS Processing
  Framework}. V úvodní kapitole je představena knihovna Qt, resp. její
verze PyQt pro jazyk Python, ve které byl celý \textbf{Workflow
  Builder} napsán. Dále je představeno prostředí QGIS a popsána práce
s QGIS Processing Frameworkem.

V druhé kapitole diplomové práce je představena samotná aplikace
Workflow Build\-er.

V poslední kapitole je zmínka o frameworku \textbf{SEXTANTE}, který se
objevil v konci psaní této práce.

\textbf{Klíčová slova}

Quantum GIS, QGIS, workflow, open source, GIS, PyQt, QGIS Processing Framework


\vfill

\section*{Abstract}
\indent The goal of this master thesis is to create a tool for Quantum GIS 
that would give user possibility to chain modules from \textbf{QGIS Processing 
Framework}. In the first chapter graphical library Qt and its binding for Python 
(PyQt) is introduced. \textbf{Workflow Builder} was written in PyQt. Quantum GIS and 
QGIS Processing Framework are also introduced.

In the second chapter of the thesis is descibed a work with the 
Workflow Builder. In the last chapter is mentioned framework \textbf{SEXTANTE} 
that appeared at the end of work on this thesis.

\textbf{Key words}

Quantum QGIS, workflow, open source, GIS, PyQt, SAGA, QGIS Processing Framework
