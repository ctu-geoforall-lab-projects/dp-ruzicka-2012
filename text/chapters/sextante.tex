\newpage
\chapter{SEXTANTE pro QGIS}
Sextante\footnote{http://www.sextantegis.com/} je svobodný GIS nástroj
psaný v jazyce Java mezi jehož základní prvky
patří \textbf{Toolbox}, \textbf{Graphical Modeler}, \textbf{batch
processing} a \textbf{správa historie}. V \textbf{Toolbox} nalezneme
všechny dostupné moduly rozdělené podle jejich
zdroje. \textbf{Graphical Modeler} umožňuje vytvářet nové moduly
spojováním již existujících modulů. Pomocí \textbf{batch processing}u
můžeme jednoduše spustit jeden modul několikrát s různými vstupními
hodnotami. \textbf{Historie} v sobě uchovává všechny procesy, které
byly provedeny. To nám umožňuje provést jakoukoliv dříve provedenou
úlohu znovu. Do historie se dále ukládají informace o chybových a
varovných hlášeních. Součástí je také řada geoalgoritmů. Jeho
implementace existují ve svobodných GIS programech psaných v Javě jako
gvSIG či OpenJUMP, a také v komerčním ArcGIS. Díky Victoru Olaya nyní
také existuje verze pro Quantum GIS, přepsaná do Pythonu.

SEXTANTE podporuje navíc parametry, jejichž implementace v QGIS
Processing Frameworku chybí. Jsou to prvky jako tabulka či seznam
vrstev. U SEXTANTE existuje více zdrojů poskytujících moduly - R,
ftools, mmqgis, Gdal a také SAGA moduly pracují lépe než u QGIS
Processing Frameworku.

\section{Srovnání s QGIS Processing Framework}

\textbf{Processing Manager} vyhledává nejen podle názvu, ale také podle tagů. To bych také uvítal i u \textbf{SEXTANTE}. Chceme-li například vypočítat akumulaci vodního toku, ale nejsme si jisti názvem modulu, u QGIS Procesing Frameworku se stačí podívat do kategorie \textit{hydrology} nebo napsat do vyhledávací řádky \textit{hydrology} a \textbf{Processing Manager} zobrazí moduly obsahující řetězec \textit{hydrology} v názvu nebo v tagu. \textbf{SEXTANTE} nám zobrazí pouze moduly, jejichž názvy obsahují řetězec \textit{hydrology}. Celkově řazení podle různých kategorií je dle mého názoru přívětivější než řazení podle zdroje (GRASS, SAGA, GDAL...). Na druhou stranu Toolbox v SEXTANTE ukládá naposledy použité moduly, což QGIS Processing Framework nedělá.

SEXTANTE podporuje daleko více modulů. Například zpřístupňuje funkce z
GRASS, SAGA, Orfeo Toolbox, GDAL či fTools. Ne všechny ale v době
psaní této diplomové práce fungovali správně.

SEXTANTE pro QGIS má propracovanější práci přes příkazovou řádku a
celkově práce s výstupními daty je jednodušší.

\section{Srovnání s Workflow Builder}

\textbf{Graphical Modeler} v SEXTANTE nabízí velmi zajímavou funkcionalitu a tou je generování kódu v Pythonu, který se generuje na základě vytvořeného workflow.

Nevýhodou je, že přidáme-li jednou algoritmus do scény, nemůžeme měnit
jeho parametry. Musíme modul smazat a znovu jej přidat. To samé platí
i u samotného spojení dvou modulů. Oproti Graphical Modeleru lze v
Workflow Builderu na první pohled vidět, kterými parametry jsou dané
moduly spojeny.
