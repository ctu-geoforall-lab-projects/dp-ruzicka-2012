\newpage
\section{Python}
\index{Python}
\nocite{py3:book}


\begin{center}
	\includegraphics[scale=0.5]{pictures/python/logo.pdf}
\end{center}

Python je objektově orientovaný a interpretovaný programovací jazyk s
dynamickým a silným typováním. Python je charakteristický pro své
vyjadřování struktury kódu pomocí odsazování, jehož dodržování je
povinné. To vede k čitelnějšímu a přehlednějšímu kódu.

První verze jazyka byla uvolněna v roce 1991. Python navrhl Guido van
Rossum, který byl inspirován jazyky jako C++ či Perl. Jedná se
o \textbf{open source} projekt dostupný pod licencí Python Software
Foundation License, která je kompatibilní s GPL licencí. Rozdíl je v
tom, že u Python Software Foundation License můžeme měnit kód bez
nutnosti zveřejnit změny jako open source
[viz \footnote{http://docs.python.org/license.html}]. Aktuální
stabilní verze jsou 2.7.3 a 3.2.3 pro verzi Python 3.0, která byla
uvolněna v roce 2008.


% noveho v python3 http://docs.python.org/release/3.0.1/whatsnew/3.0.html

V dnešní době se s Pythonem můžeme setkat téměř všude. Pro jeho
jednoduchost při psaní kódu je velmi populární a široce
rozšířený. Jeho velká výhoda je tedy velmi čitelný kód a rychlost
psaní.
% jazyk je prenositelny? zminka o "Compiled" Python files?
Jazyk je jednoduše přenositelný (není nutná kompilace), je multiplatformní a má kvalitní dokumentaci. 

V mnoha projektech existuje skriptovací rozhraní pro Python. Jako
příklad uveďme QGIS, GIMP, Inkscape, Scribus, LibreOffice, Blender
nebo ArcGIS. Můžeme jej najít v projektech jako Maya, OpenShot Video
Editor, Wammu, DopBox, MapServer či Gajim. Používá se pro psaní
grafického rozhraní. Existují verze grafických knihoven Qt a GTK pro
Python - PyQt, resp. PyGTK. Využívá se jako skriptovací jazyk pro
psaní webových aplikací. V síti Internet se s ním můžeme také setkat v
podobě nástrojů jako Django, Zope či Pylons. Byly napsány knihovny pro
vědecké výpočty - NumPy, SciPy, Matplotlib.

%Řekl bych, že odinstalovat Python z Vašeho počítače nebude zrovna ten nejlepší nápad. Je totiž prakticky všude.

\subsection*{PyQGIS}
Quantum GIS nabízí podporu QGIS API pro jazyk Python. Jedná se o verzi
PyQGIS. PyQGIS můžeme používat přímo v QGISu přes příkazovou řádku,
psát zásuvné moduly či využít QGIS API pro náš vlastní program.

Existuje velmi dobře zpracovaná tzv. kuchařka jak psát v
PyQGIS \cite{pyqgis:www}, tudíž nepovažuji za nutné se mnoho
rozepisovat. Zmíním jen pár informací, které považuji za
základní. Rastrové a vektorové vrstvy jsou reprezentovány
třídami \textit{QgsRaster\-Layer}
a \textit{QgsVectorLayer}. Třída \textit{QgsMapLayerRegistry} slouží k
načítání, správu a mazání geografických vrstev z QGISu. Třída se
nachází v \textbf{qgis.core} knihovně. Pomocí
příkazu \textbf{QgsMapLayer.instance().mapLayers()} získáme všechny
vrstvy momentálně načtené v QGIS. Aktivní vrstvu získáme pomocí
příkazu \textbf{qgis.utils.iface.activeLayer()}. Vrstvu přidáme do
QGISu pomocí \textbf{QgsMapLayer.instance().addMapLayer(layer)} a
odebereme \textbf{QgsMapLayer.instance().removeMapLayer(layer)}.

Vektorovou vrstvu vytvoříme
takto \textbf{QgsVectorLayer}(cesta\_k\_souboru, jmeno\_vrstvy,
knihovna).

Například:

\begin{lstlisting}
	vector = QgsVectorLayer("~/geodata/contour.shp", "ContourLines", "ogr")
\end{lstlisting}

Rastrovou vrstvu vytvoříme takto \textbf{QgsRasterLayer}(cesta\_k\_souboru, jmeno\_vrstvy).

Například:

\begin{lstlisting}
	raster = QgsRasterLayer("~/geodata/elevation.tiff", "Elevation")
\end{lstlisting}

Z mapových vrstev můžeme získat například souřadnicový systém pomocí
metody \textit{crs}(), zdrojový soubor metodou \textit{source}(),
jménou \textit{name}() či rozsah \textit{exent}().\\

Kompletní dokumentace QGIS API je dostupná zde \cite{qgis_api:www}.
%QGIS používá verzi 2.7.1.

%{\color{red}napsat vie o PyQGIS}






