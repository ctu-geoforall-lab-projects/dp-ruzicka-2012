\newpage
\section{Třídy}

O logickou část Workflow Builderu se starají třídy Graph
reprezentující graf, SubGraph reprezentující podgraf, Module
reprezentující modul, Connection reprezentující spojení a Port
reprezentující parametr modulu. Na diagramu \figurename \ref{logClass}
je znázorněný vztah po vytvoření podgrafů (souvislých komponent) v
grafu. Existuje vždy právě jedna instance třídy \textbf{Graph}. Tato
instance může obsahovat libovolné množství instancí
třídy \textbf{SubGraph}. Každá tato instance obsahuje minimálně jednu
instanci třídy \textbf{Module}, dále může obsahovat instance
třídy \textbf{Connection}. \textbf{Module} může obsahovat instance
tříd \textbf{Port}, \textbf{Connection} a právě jednu instanci
třídy \textbf{SubGraph}. Spojení v sobě drží informaci o počátečním a
koncovém parametru a modulu.

\begin{figure}[h]
	\begin{center}
		\includegraphics[scale=0.05,angle=-90]{pictures/wf/logClass.pdf}
		\caption{Diagram znázorňující vztahy mezi třídami Graph, SubGraph, Connection, Module a Port}
                % zdroj ? (ML)
	  	\label{logClass}
	\end{center}
\end{figure}

Dialogové okno je instance třídy \textbf{WorkflowBuilder}, která je
potomkem třídy \textbf{QDialog} z knihovny Qt. WorkflowBuilder se
skládá z \textbf{GraphicsView} (reimplementace třídy QGraphicsView z
Qt). GraphicsView zobrazuje prvky skrze scénu (\textbf{DiagramScene} -
potomek QGraphicsScene z Qt). Třídu Module reprezentuje ve scéně
třída \textbf{QGraphicsModuleItem}, třídu Connection
třída \textbf{QGrahicsConnectionItem} a parametry jsou
reprezentovány \textbf{QGraphcisPort}.

\begin{figure}[h]
	\begin{center}
		\includegraphics[scale=0.1,angle=-90]{pictures/wf/graphClass.pdf}
		\caption{Diagram znázorňující vztahy mezi třídami GraphcisView, GraphicsScene, QGraphicsModuleItem, QGraphicsConncetionItem a QGraphicsPortItem v třídě WorkflowBuilder}
                % zdroj ? (ML)
	  	\label{graphClass}
	\end{center}
\end{figure}

\newpage
\subsection*{Třída Graph}
Třída Graph je v podstatě samotné workflow. Obsahuje všechny moduly a
spojení, které se ve workflow vyskytují. Hlavní metody jsou
$executeGraph$() a $save$(). Metoda $executeGraph$() postupně prochází
všechny podgrafy a jsou-li validní a neobsahují cyklus, spouští jejich
moduly. Validní podgraf je ten, u jehož každého modulu jsou všechny
vstupní parametry buď nastaveny nebo spojeny s jiným. Metoda $save$()
vytvoří xml soubor reprezentující nový modul a obsahující všechny
podgrafy, moduly a spojení. Metoda $addConnection$() přidá do grafu
spojení, $addModule$() přidá do grafu modul, $addSubGraph$() přidá do
grafu podgraf, $findLoop$() prochází graf a vrací True, najde-li v
grafu cyklus, $xml$() vytvoří DOM reprezentaci grafu.

\subsection*{Třída SubGraph}
V řeči teorie grafů instance třídy \textbf{SubGraph} reprezentují
souvislé komponenty grafu. V našem případě se jedná o instanci
třídy \textbf{Graph}, která reprezentuje workflow.

Hlavní metody jsou $executeSGraph$() a $xml$(). Metoda
$executeSGraph$() spouští všechny moduly v podgrafu. Metoda $xml$() je
důležitá při ukládání nového modulu do souboru, vytvoří DOM
reprezentaci podgrafu. Metody $prepareToExecute$() a $findLoop$() se
spouští před samotným spuštěním podgrafu. Metoda $prepareToExecute$()
prochází všechny moduly a zjišťuje, zdali jsou u každého modulu
nastaveny vstupní parametry či jsou spojeny s jiným
parametrem. Pakliže jsou, označí podgraf jako validní pomocí metody
$setValid$(). Metoda $findLoop$() slouží k nalezení cyklu v daném
podgrafu.
%processing.framework[self.label].instance()
Pomocí metod $addModule$() a $setConnections$() přidáme do podgrafu modul, resp. nastavíme spojení.

\subsection*{Třída Module}
Třída Module reprezentuje PF Module v prostředí Workflow
Builderu. Instance třídy v sobě uchovávají jméno, popis, tagy a
parametry PF Modulu. Parametry se uchovávají v podobě instance třídy
Port.

Důležité jsou metody $getInstancePF$(), $execute$() a $xml$(). Metoda
$getInstancePF$() vrací již nastavenou instanci třídy PF Module, která
koresponduje s modulem z Workflow Builderu. Pakliže u modulu ještě
nebyla vytvořena instance PF Modulu, vytvoří ji pomocí
$processing.framework[nazev\_modulu].instance()$. \textbf{Port}ům
modulu nastaví odkazy na parametry právě vytvořené instance třídy PF
Module.

Metoda $execute$() nastaví instanci PF Modulu parametru podle
aktuálních hodnot \textbf{Port}ů modulu a spustí instanci PF
Modulu. Potom nastaví hodnoty výstupů z PF Modulu do \textbf{Port}ů
modulu a dále nastaví hodnoty i u \textbf{Port}ů, které jsou s daným
výstupem (Portem, parametrem) spojené.

Metoda $xml$() vytvoří DOM reprezentaci Modulu, která slouží pro
uložení celého workflow do souboru formátu xml.

Instance třídy Module je jednoznačně identifikovatelná pomocí jejího
identifikačního čísla, které je v rámci grafu (Graph) jedinečné.

Instance třídy \textbf{Module} jsou ve scéně reprezentována instancemi
třídy \textbf{QGraphicsModelItem}.

\subsection*{Třída Port}  
Instance třídy \textbf{Port} reprezentují parametry PF Modulu. Jsou
jednoznačně identifikovatelné pomocí identifikačního čísla, které je v
rámci modulu jedinečné a pomocí identifikačního čísla modelu.

Uchovává v sobě informace jako název parametru, typ, zdali je parametr
volitelný či povinný, zdali je parametr výstupní či vstupní, popis
nebo výchozí hodnotu. Po úspěšném spuštění modulu a v případě, že
je \textbf{Port} výstup, uloží se také nová hodnota.

Pomocí metody $getValue$() získáme aktuální hodnotu, metoda
$outputData$() vrací výstupní data, $destinationPorts$() vrací porty,
které jsou s daným portem spojené a ve spojení jsou vedeny jako
cílové, $getToolTip$() vrací textový řetězec sloužící jako nápověda
pro daný port, $isConnected$() vrací zdali je daný port spojen s jiným
a $xml$() vrací DOM reprezentaci portu. Je-li \textbf{Port} výstupní,
pomocí metody $addItToCanvas$() zjistíme, zdali si uživatel přál
načíst vrstvu po spuštění modulu do QGIS, a metoda $outputName$() nám
vrátí jméno, pod kterým se má vrstva načíst.

Instance třídy \textbf{Port} jsou ve scéně reprezentovány instancemi
třídy \textbf{QGraphicsPortItem}.

\subsection*{Třída Connection}
Třída \textbf{Connection} v terminologii teorie grafů reprezentuje
hrany. Uchovává v sobě informaci o počátečním a koncovém modulu
(Module), resp. parametru (Port). A obsahuje jedinou metodu xml(),
která vrací DOM reprezentaci spojení.

Instance třídy \textbf{Connection} jsou ve scéně reprezentovány
instancemi třídy \textbf{QGraphicsConnectionItem}.

\subsection*{Třída GraphicsView}
Třída \textbf{GraphicsView} je reimplementací
třídy \textbf{QGraphicsView} z knihovny Qt. Byla reimplementována
metoda $wheelEvent$(), která umožňuje funkci zoom, a metody
$dragEnterEvent$(), $dragMoveEvent$() a $dropEvent$() pro spravování
událostí týkajících se prostředí Drag and Drop. \textbf{GraphicsView}
přijímá pouze objekty z Processing Manageru. Metoda $keyPressEvent$()
je reimplementována tak, aby se po stisknutí klávesy $Delete$ smazaly
všechny vybrané prvky.


\subsection*{Třída DiagramScene}
Třída \textbf{DiagramScene} je reimplementací
třídy \textbf{QGraphicsScene} z knihovny Qt. Byly reimplementovány
metody \textit{mousePressEvent}(), \textit{mouseMoveEvent}()
a \textit{mouseReleaseEvent}(). Tyto metody řeší, zdali uživatel pouze
kliknul na modul a chce, aby se mu zobrazili informace o parametrech,
či kliknul na parametr a chce jej spojit s jiným. Také se zde řeší,
zdali mohou být parametry spojeny. Pakliže ano, vytvoří se spojení
(instance třídy Connection) a na jeho základě také instance
třídy \textbf{QGraphicsConnection}.

Pomocí metod \textit{addModule}(processing.Module) se vytvoří nejdříve
Objekt třídy Module a na jeho základě
objekt \textbf{QGraphicsModuleItem}. Metoda \textit{delModule}() maže
modul ze scény (DiagramScene) i z grafu (Graph) a zároveň i jejich
spojení s druhými moduly. Metoda \textit{delConnection}() maže spojení
ze scény (DiagramScene) i z grafu (Graph).

Metoda \textit{clearDockPanel}() smaže informace z pravého postranního
panelu.
