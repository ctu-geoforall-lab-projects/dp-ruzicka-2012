\newpage

\section{xml.dom.monidom}
\nocite{python:www}
\nocite{py3:book}

Nově vzniklé workflow se ukládají do souboru využívajíce
jazyk \index{XML} XML. Výhoda XML dokumentu je, že se s ním snadno
pracuje a jde v podstatě pouze o textový dokument.

XML nabízí jednoduché uložení hierarchicky strukturovaných dat. O
prvcích XML dokumentu hovoříme jako o elementech. Elementy jsou
ohraničeny počátečními a koncovými znaky, tzv. tagy. XML dokument
obsahuje vždy právě jeden kořenový element, který se může skládat z
dalších a dalších elementů. V příkladu XML dokumentu
(\autoref{xml:example}]) je kořenový element $Graph$. Elementy mohou
obsahovat atributy (dvojice jméno="hodnota").  Jména atributů se v
rámci jednoho elementu nesmí opakovat. Elementy také mohou obsahovat
text, který se uvádí mezi počátečním a koncovým znakem. \\

\noindent Příklad XML dokumentu$:$ \\

\lstinputlisting[language=XML,caption={Příklad XML dokumentu},morekeywords={Graph, SubGraph, Module, tag}, label=xml:example]{code/plugins/xml/example_xml.xml}

% DOM level 2
Pro práci s XML dokumenty se v prostředí jazyka Python nabízí několik
modulů. Při psaní této práce byl
vybrán \index{xml.dom} \textbf{xml.dom}, resp. jeho odlehčená
verze \index{xml.dom.minidom} \textbf{xml.dom.minidom}, který k XML
dokumentu přistupuje přes rozhraní DOM (Document Object
Model). \index{DOM} \textbf{DOM} je rozhraní pro přístup a práci s XML
dokumenty. Je to zároveň standard organizace World Wide Web Consortium
(W3C). W3C je organizace, která se zabývá standardy v prostředí Word
Wide Web.

Na začátku je potřeba si vytvořit objekt (\textit{Document}), který
bude reprezentovat celý XML dokument. \textit{Document} je, stejně
jako všechny ostatní elementy (objekty třídy Element) XML dokumentu,
podtřídou třídy \textit{Node}. Do takto vytvořeného objektu se poté
mohou přidávat další elementy. Třída \textbf{Document} obsahuje
statickou metodu $createElement$(\textit{tagName}). Metodu můžeme tedy
volat z třídy \textbf{Document} nebo z již existujícího objektu. To
samé platí i v případě metody createTextNode(), která slouží k
vytvoření textu, který se poté může vložit do elementu.

Pro přidání elementu do jiného elementu použijeme metodu
$appendChild$(\textit{newChild}) nebo
$insertBefore$(\textit{newChild}, \textit{refChild}). Chceme-li
nahradit jeden element druhým, použijeme metodu
$replaceChild$(\textit{newChild}, \textit{oldChild}). Pro mazání
elementu se používá metoda $removeChild$(\textit{oldChild}). K získání
informace, zdali element obsahuje atributy, zavoláme metody
$hasAttributes$(). Všechny tyto metody jsou metody
třídy \textbf{Node}. Třídy jako \textbf{Document} či \textbf{Element},
stejně jako všechny ostatní prvky XML dokumentu, jsou podtřídami
třídy \textbf{Node}, tudíž i ony dědí tyto metody.

Atributy nastavíme u objektů třídy \textbf{Element} pomocí metody
$setAttribute$(\textit{name}, \textit{value}). Metodou
$hasAttribute$(\textit{name}) se dotazujeme, zdali daný element
obsahuje atribut \textit{name}. Pomocí metody
$getAttribute$(\textit{name}) získáme hodnotu
atributu \textit{name}. A pomocí metody $removeAttribute$(name)
smažeme atribut \textit{name}.

Metoda $getElementsByTagName$(\textit{tagName}) vrací seznam elementů
korespondující s \textit{tagName} a potomky daného elementu
nacházející se v daném elementu.

XML dokument uložíme do souboru zavoláním metody
writexml(\textit{file}, \textit{indent}, \textit{addindent}, \textit{encoding})
dané instance třídy Document. \textit{File} je soubor připravený pro
zápis, \textit{indent} udává odsazení na začátku nového elementu
(například začátek nového řádku), \textit{addindent} udává přírůstkové
odsazení pro potomky daného elementu (například tabulátor)
a \textit{encoding} je kódování. Pouze \textit{file} je povinný
parametr.

Příklad z [\autoref{xml:example}] bychom tedy mohli vytvořit kódem
[\autoref{xml:xml}] a zapsat do souboru
[\autoref{xml:save}]. {\color{red}upravit}\\

\newpage
\begin{lstlisting}[label=xml:xml,caption={ukázka tvorby XML dokumentu z [\autoref{xml:example}]}, morekeywords={Document, createElement, setAttribute, createTextNode, addChild}]
from xml.dom.minidom import Document

doc = Document()

graph = Document().createElement("Graph")
graph.setAttribute("name","Addition two rasters")
doc.addChild(graph)

graphDesc = Document().createTextNode("Description...")
graph.addChild(graphDesc)

subGraph = doc.createElement("SubGraph")
subGraph.setAttribute("id","17")
graph.addChild(subGraph)

module01 = doc.createElement("Module")
module01.setAttribute("id","769")
module01.setAttribute("name","Input raster by path")
module01Desc = doc.createTextNode("You can register ... the path.")
module01.addChild(module01Desc)
tag = doc.createElement("Tag")
tagDesc = doc.createTextNode("workflow builder")
module01.addChild(tag)
subGraph.addChild(module01)

module02 = doc.createElement("Module")
module02.setAttribute("id","998")
module02.setAttribute("name","Operations with two rasters")
module02Desc = doc.createTextNode("Pixel by pixel ... rasters.")
module02.addChild(module02Desc)
subGraph.addChild(module02)
\end{lstlisting}

\newpage
\begin{lstlisting}[label=xml:save,caption={uložení XML dokumentu do souboru}]
from xml.dom import Document

file = open("xml.xml", "w")

doc = Document()
doc.writexml(file, indent="\n",  addindent="\t", encoding="UTF-8")

file.close()

\end{lstlisting}

Pro otevření souboru použijeme metodu $xml.minidom.parse$(\textit{path}), kde \textit{path} je cesta k xml souboru.

%Node - zakladni element, od kterého všechny elementy dědí
%Document
% - createElement(tagName="Graph") - vraci Element
% - createTextNode("Description...")
% - writexml(f,  indent="\n",  addindent="\t", encoding="UTF-8") - writexml(writer[, indent=""[, addindent=""[, newl=""]]]) - writer = soubor otevřený pro zapisování, indent="\n" - znamená, odsazení nového node-uzle-, addindent="\t" - znamená, jak se bude navyšovat odsazení pří zanořování ve stromě - v tomto případě se vždy přidá tabulátor, encoding = kódování
%Element
% - 	setAttribute(name="name", Addition two rasters)
% - 	appendChild( ) napr. text node ci dalsi Element	
%Attr
%
%Příklad:
%otevřu soubor pro čtení
%vytvořím Document()
%pár prvků, jim atributy a text
%uložím

%\begin{table}	
%	\centering
%	\begin{tabular}{|c|c|}
%		\hline
%		atribut & příklad \\
%		\hline
%		name & Addition two rasters \\
%		tags & ['raster', 'hydrology'] \\	
%		\hline	
%	\end{tabular}
%	\caption{atributy elementu Graph}
%	\label{tab:graph}
%\end{table}
