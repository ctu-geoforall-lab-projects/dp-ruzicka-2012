\section*{Abstrakt}\label{abstract}
\indent Diplomová práce si vytyčila za cíl vytvořit v prostředí Quantum GIS ( QGIS ) nástroj, který by umožňoval uživateli grafické propojování modulů z frameworku \textbf{QGIS Processing Framework}. V úvodních dvou kapitolách je představeno prostředí QGIS a popsána práce s QGIS Processing Frameworkem jak z pozice uživatele GISu, tak z pozice vývojáře nových modulů. Dále je zmíněn SAGA Plugin, který byl napsán v rámci onoho frameworku.

V dalších částech diplomové práce představím z rychlíku knihovnu Qt, resp. její verzi PyQt pro jazyk Python, ve kterém byl celý Workflow Builder napsán.

V poslední kapitole diplomové práce je představena samotná aplikace umožňující grafické propojování modulů z QGIS Processing Framework Manageru do grafu, jejich postupné spuštění a případné uložení jako nový modul.

\textbf{Klíčová slova}

Quantum QGIS, workflow, open source, GIS, PyQt, SAGA, QGIS Processing Framework


\vfill

\section*{Abstract}
\indent

\textbf{Key words}

Quantum QGIS, workflow, open source, GIS, PyQt, SAGA, QGIS Processing Framework