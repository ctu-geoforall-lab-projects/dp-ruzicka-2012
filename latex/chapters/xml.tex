\newpage
\section{.xml.dom.monidom}
\nocite{python:www}
\nocite{py3:book}

Nově vzniklé workflow se ukládají ve formátu \index{XML} XML. Výhoda tohoto formátu je, že se s ním snadno pracuje a jde v podstatě pouze o textový dokument.

Python nabízí několik možností pro zpracování XML dokumentů.

XML nabízí jednoduché uložení hierarchicky strukturovaný dat. O prvcích XML dokumentu hovoříme jako elementech. Elementy jsou ohraničeny počátečními a koncovými znaky, tzv. tagy. XML dokument obsahuje vždy právě jeden kořenový element. Ten se může skládat z dalších a dalších elementů. Elementy mohou obsahovat atributy (dvojice jméno="hodnota").  Atributy se nesmějí v opakovat. Elementy kromě toho mohou obsahovat text. Text se uvádí mezi počáteční a koncový znak. \\

\noindent Příklad XML dokumentu$:$ \\

\lstinputlisting[language=XML,caption={Příklad XML dokumentu},morekeywords={Graph, SubGraph, Module, tag}, label=xml:example]{code/plugins/xml/example_xml.xml}

Pro práci s XML dokumenty existuje v prostředí jazyka Python několik modulů. Mě se osvědčil modul xml.dom.minidom. Na začátku je potřeba si vytvořit objekt (\textit{Document}), který bude reprezentovat celý XML dokument. \textit{Document} je, stejně jako všechny ostatní elementy XML dokumentu, je podtřídou třídy \textit{Node}. Do takto vytvořeného objektu se poté mohou

%Node - zakladni element, od kterého všechny elementy dědí
%Document
% - createElement(tagName="Graph") - vraci Element
% - createTextNode("Description...")
% - writexml(f,  indent="\n",  addindent="\t", encoding="UTF-8") - writexml(writer[, indent=""[, addindent=""[, newl=""]]]) - writer = soubor otevřený pro zapisování, indent="\n" - znamená, odsazení nového node-uzle-, addindent="\t" - znamená, jak se bude navyšovat odsazení pří zanořování ve stromě - v tomto případě se vždy přidá tabulátor, encoding = kódování
%Element
% - 	setAttribute(name="name", Addition two rasters)
% - 	appendChild( ) napr. text node ci dalsi Element	
%Attr
%
%Příklad:
%otevřu soubor pro čtení
%vytvořím Document()
%pár prvků, jim atributy a text
%uložím

%\begin{table}	
%	\centering
%	\begin{tabular}{|c|c|}
%		\hline
%		atribut & příklad \\
%		\hline
%		name & Addition two rasters \\
%		tags & ['raster', 'hydrology'] \\	
%		\hline	
%	\end{tabular}
%	\caption{atributy elementu Graph}
%	\label{tab:graph}
%\end{table}

načtení xml

-------------


