\newpage
\section{Python}
\index{Python}
\nocite{py3:book}


\begin{center}
	\includegraphics[scale=0.5]{pictures/python/logo.pdf}
\end{center}

Python je objektově orientovaný a interpretovaný programovací jazyk s dynamickým a silným typováním. Python je charakteristický pro své vyjadřování struktury kódu pomocí odsazování, jehož dodržování je povinné. To vede k čitelnějšímu a přehlednějšímu kódu. 

% dynamické typování -> pomalejší x staticke - typ promenne se prirazuje uz pri kompilaci, jsme vazani na typ
% silne typovani - nemuzeme scitat hrusky s jabkami

První verze jazyka byla uvolněna v roce 1991. Python navrhl Guido van Rossum, který byl inspirován jazyky jako C++ či Perl. Jedná se o open source projekt dostupný pod licencí Python Software Foundation License, která je kompatibilní s GPL licencí. Rozdíl je v tom, že můžeme měnit kód bez nutnosti zveřejnit změny jako open source [viz \footnote{http://docs.python.org/license.html}. Aktuální stabilní verze jsou 2.7.3 a 3.2.3 pro verzi Python 3.0, která byla ulovněna v roce 2008. 


% noveho v python3 http://docs.python.org/release/3.0.1/whatsnew/3.0.html

V dnešní době se s Pythonem můžeme setkat doslova všude. Pro jeho jednoduchost při psaní kódu je velmi populární a široce rozšířený. Jeho veliká výhoda je tedy velmi čitelný kód a rychlost psaní. Jazyk je jednoduše přenositelný (není nutná kompilace), je multiplatformní a má kvalitní dokumentaci. 

V mnoha projektech existuje skriptovací rozhraní pro Python. Jako příklad uveďme QGIS, GIMP, Inkscape, Scribus, LibreOffice, Blender nebo ArcGIS. Můžeme jej najít v projektech jako Maya, OpenShot Video Editor, Wammu, DopBox, MapServer či Gajim. Používá se pro psaní grafického rozhraní. Existují verze grafických knihoven Qt a GTK pro Python - PyQt, resp. PyGTK. Využívá se jako skriptovací jazyku pro psaní webových aplikací. V síti Internet se s ním můžeme také setkat v podobě nástrojů jako Django, Zope či Pylons. Byly napsány knihovny pro vědecké výpočty - NumPy, SciPy, Matplotlib.

Řekl bych, že odinstalovat Python z Vašeho počítače nebude zrovna ten nejlepší nápad. Je totiž prakticky všude.

\subsection*{PyQGIS}
Quantum GIS nabízí podporu skriptování v jazyce Python. Jedná se o verzi PyQGIS. PyQGIS můžeme používat přímo v Quantum GISu přes příkazovou řádku, můžeme psát zásuvné moduly pro QGIS či využít QGIS API pro náš vlastní program. 

%QGIS používá verzi 2.7.1.

{\color{red}napsat vie o PyQGIS}






