\newpage
\section{Spuštění workflow}
\nocite{demel:book}
Workflow se spouští tlačítkem $Execute$. Jakmile se uživatel rozhodne spustit celý proces (workflow), v grafu se vytvoří podgrafy. Ty jsou v podstatě souvislými komponentami grafu. Tvoří se rekurzivně tak, že se vytvoří podgraf třídy \textbf{SubGraph} a z prvotního seznamu všech modulů v grafu se vyjme jeden a vloží se do něj. Poté se hledají další moduly, které patří do stejné komponenty, do stejného podgrafu. Z původního seznamu všech modulů v grafu se vyjmou všechny moduly spojené s prvním modulem a uloží se do podgrafu, poté se z původního seznamu vyjmou moduly, které jsou spojené s předchozími moduly a tak dále dokud existují spojení. Zároveň se ukládají do podgrafu i spojení (instance třídy Connection). Pakliže již neexistuje další propojený modul a v původním seznamu ještě zůstali nějaké moduly, vytvoří se nový podgraf a postupuje se stejným způsobem jako u předchozího podgrafu. Jeli původní seznam modulů prázdný, znamená to, že všechny moduly z grafu jsou rozděleny do podgrafů.

Jakmile máme vytvořeny podgrafy, zjistíme, zdali je pro nás graf, resp. všechny jeho podgrafy, validní. To znamená, že se prochází každý podgraf a u každého modulu se kontroluje, zdali jsou u jeho modulů nastaveny všechny povinné vstupní parametry, případně jestli u nich existuje spojení. Pakliže se narazí na modul, u kterého není nějaký povinný vstupní parametr nastaven, uloží se do seznamu nevalidních modulů. Projdou-li se všechny moduly v podgrafu a alespoň jeden není v pořádku, není validní, vypíše se hlášení na lištu ve spodní části Workflow Builderu s informací, že některé moduly nejsou nastaveny a nemůže se pokračovat ve spouštění workflow. Zároveň se také označí moduly, o které se jedná.

Jsou-li všechny povinné vstupní parametry u všech modulů nastaveny nebo spojeny s jiným, zkontroluje se každý podgraf, zdali neobsahuje cyklus. To se provádí pomocí prohledávání grafu do hloubky [viz \lstlistingname \ref{findLoop}]. Neprojde-li kontrola, vypíše se hláška, že graf obsahuje cyklus. Projde-li kontrola a graf neobsahuje cykly, začnou se spouštět postupně všechny podgrafy. Tím zjistíme, že jsou validní a neměli by se objevit žádné známé problémy. 


\newpage
\begin{lstlisting}[label=findLoop,caption={Hledání cyklu v podgrafu},morekeywords={find}]
        def find(v):
        	# oznacim si vrchol
            v.mark = True
            eV = [seznam hran vychazejicich z vrcholu v]            
            for e in eV:
                w = e[1] # koncovy modul spojeni
                # jedna se o puvodni vrchol
                if w.id is vv.id:
                    vv.loop =  True
                    break
                # prozkoumame ho
                if not w.mark:
                    find(w)
        
        V = [seznam vrcholu v podobe Module]
        E = [seznam dvojic pocatecni a koncovy modul spojeni]

        # prochazim vrchol podgrafu
        for vv in V:
            find(vv)
            for v in V:
                v.mark = False
            if vv.loop:
            	# najdu-li cyklus
                return vv.loop

        return False
\end{lstlisting}

Samotné spouštění podgrafu začne tak, že se vezme libovolný modul z podgrafu a zkontroluje se, zdali jsou nastaveny všechny vstupy. Pakliže jsou všechny nastaveny, vytvoří se instance PF Modulu, nastaví se parametry a spustí se. Jsou-li výstupní parametry spojeny s jinými moduly, nastaví se hodnota parametru na druhém konci spojení právě získanou hodnotou. Pakliže nejsou některé vstupní parametry nastaveny, sleduje se jejich spojení a pokusí se spustit předchozí modul. Pakliže i u něho jsou nějaké parametry nenastaveny, opět se sleduje jejich spojení. Vše se opakuje do té doby, dokud se nespustí nějaký modul a ten po úspěšném provedení nastaví vstupní hodnoty v grafu následujících modulů na základě svých výstupních hodnot. Postupně se nakonec spustí všechny moduly a výstupní data se uloží. Tento proces se také řeší rekurzí.

Pozn. kontrolují se pouze vstupní parametry, protože SAGA Plugin momentálně ignoruje, zdali nastavíme výstupní parametr či ne - vytvoří si vždy nový.