\newpage
\chapter{Workflow Builder}

\section{Tvorba workflow}
\section{Uložení workflow}
\subsection{Popsání výstupního xml souboru}
XML nabízí jednoduché uložení hierarchicky strukturovaný dat. O prvcích XML dokumentu hovoříme jako elementech. Elementy jsou ohraničeny počátečními a koncovými značkami, tzv. tagy. XML dokument obsahuje vždy právě jeden kořenový element. Ten se může skládat z dalších a dalších elementů. V našem případě je kořenový element Graph. Ten se skládá z minimálně jednoho podgrafu (SubGraph), a ten poté minimálně z jednoho modulu (Module). Podgraf dále může obsahovat spojení mezi moduly (Connection). Modul kromě toho obsahuje elementy parametr (Port) a tag (tag) a popis. Graf také obsahuje tagy a popis. 

\begin{table}	
	\centering
	\begin{tabular}{|c|c|}
		\hline
		atribut & příklad \\
		\hline
		name & Addition two rasters \\
		tags & ['raster', 'hydrology'] \\	
		\hline	
	\end{tabular}
	\caption{atributy elementu Graph}
	\label{tab:graph}
\end{table}

\section{Načtení workflow do PF Manageru}
