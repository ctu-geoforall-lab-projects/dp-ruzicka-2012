%& --translate-file=il2-cs
\documentclass[a4paper,12pt,oneside,notitlepage,openright]{report} %print on both side

%%%%%%%%%%%%%%%%%%%
%%%% PREAMBULE %%%%
%%%%%%%%%%%%%%%%%%%
\usepackage[utf8]{inputenc} 		% neco kvuli cestiny :) \
\usepackage[IL2]{fontenc}			% neco kvuli cestiny :)  | - nevim, co na to windows :)
\usepackage[czech]{babel}   		% neco kvuli cestiny :) /
\usepackage{a4wide} 				% aby byl text sirsi ... vpodstate tedy normalni
\usepackage{indentfirst}			% pro odsazeni prviho odstavce

%%%%%% Printing code
%\usepackage[scaled]{beramono}
%\renewcommand*\familydefault{\ttdefault}
\usepackage{listings}
\renewcommand\lstlistingname{Ukázka kódu}
\renewcommand\lstlistlistingname{Ukázky kódu}
%The first changes the caption name for listings. The second — the header name for the list of listings which is printed by the \lstlistoflistings command.
\def\lstlistingautorefname{Uk.}

\lstset{
  language=Python,
  basicstyle=\footnotesize,
  numberstyle=\tiny\color{gray},
  frame=single,
  keywordstyle=\color{black}\textbf,
  showstringspaces=false,
  captionpos=b,
  formfeed=\newpage,
  tabsize=3,
  commentstyle=\itshape,
  stringstyle=\ttfamily,
  numbers=left
  %morekeywords={models, lambda, forms}
}


%%%%%% Literature
\usepackage{biblatex}
\bibliography{thesis}

%%%%%% tady jsem nastavoval okraje pro tisknuti, snad to tak muze byt :)
\setlength{\oddsidemargin}{0mm} 
\setlength{\textwidth}{155mm}\setlength{\topmargin}{0mm} 
\setlength{\textheight}{230mm}
\setlength{\headheight}{14.5pt}         %height of head
\setlength{\oddsidemargin}{7mm}         %odd left margin
\setlength{\evensidemargin}{-\oddsidemargin} %even left margin

% balicek pro tvorbu indexu - myslim, ze je zde standardne
\usepackage{index}	% pro tvorbu rejstriku
\newindex{default}{idx}{ind}{Rejstřík}

% to je neco pro tvorbu odkazu a nacitani obrazku
\usepackage[pdftex]{hyperref}
\usepackage[pdftex]{graphicx}
\renewcommand{\figurename}{Rajah}

% barevny text
\usepackage{color}
\definecolor{gray}{rgb}{0.5,0.5,0.5}

\def \u{1.5}

%%%% zahlavi, zapati .... da se s tim pekne pohrat
\usepackage{fancyhdr}
\fancyhf{}
\fancyfoot[C]{\thepage} 
\fancyhead[L]{\slshape \leftmark}
\renewcommand{\headrulewidth}{0.4pt} 
\renewcommand{\footrulewidth}{0.0pt}    


% nastaveni radkovani na 1.5 radku
\def\baselinestretch{1.5}\normalsize
\begin{document}
%-----<<< TITULNI STRANA >>>-----
%-----<<< TITLE PAGE >>>-----
\def \CVUT {České vysoké učení technické v~Praze}
\def \FSV {Fakulta stavební}
\def \KATEDRA {Katedra mapování a kartografie}
\def \BookName {DIPLOMOVÁ PRÁCE}
\def \Bookname {Workflow builder pro Quantum GIS}
\def \Authors {Zdeněk Růžička}
\def \DatumDP {Praha, 2012}
\newcommand{\vedouci}{Ing. Martin Landa}

\def \BookHeadDP {
    \begin{center}
    {\LARGE \textsc \CVUT}\\[2ex]
    {\LARGE \textsc \FSV}\\

    \vspace{2cm}
	\begin{center}
		\includegraphics[scale=0.55]{pictures/lev}
	\end{center}
    \vspace{1cm}

    {\textbf {\Huge \BookName \\[4ex]}
    {\LARGE \bf \Bookname}
    }

   \vfill
   {\large
    \begin{tabular}{rl}
    Vypracoval: & \Authors\\
    Vedoucí práce: & \vedouci\\
    Rok: & \DatumDP
    \end{tabular}
   }

    \newpage
    \end{center}
}
\pagestyle{empty}                       %no pagination
\BookHeadDP
\cleardoublepage

%-----<<< ---------- >>>-----
%%-----<<< ---------- >>>-----
%
%%-----<<< PROHLASENI >>>-----
\pagestyle{plain}                       %no pagination
\pagenumbering{roman}                   %start roman pagination from 1
\setcounter{page}{2}
\vfill

\vglue 13cm

\section*{Prohlášení}
\indent\indent Prohlašuji, že jsem svou diplomovou práci na téma \textbf{Workflow Builder} vypracoval samostatně s pomocí svého vedoucího práce a za použití literatury a zdrojů uvedených v~přiloženém seznamu na konci práce.
\vspace{0.5cm}

%\iffalse
%\noindent Nemám závažný důvod proti užití  tohoto
%školního díla ve smyslu § 60 Zákona č.121/2000 Sb. , o~právu
%autorském, o~právech souvisejících s~právem autorským a o~změně
%některých zákonů (autorský zákon). {\em Tento odstavec je
%nepovinný, záleží na vašem rozhodnutí.}
%\fi

\begin{table}[!hbp]
\begin{tabular}{lr}
  V~Praze dne
  \begin{tabular}{p{2.5cm}}
    \\
    \hline
  \end{tabular}
 &
 \hspace{4cm}
 \begin{tabular}{p{3.5cm}}
    \vspace{0.6cm} \\
     \hline \hspace{1cm} podpis
 \end{tabular}

\end{tabular}
\end{table}

\newpage
%%-----<<< ---------- >>>-----
%
%%-----<<< PODEKOVANI >>>-----
\vfill

\vglue 14cm

\section*{Poděkování}
\indent\indent Především děkuji vedoucímu mé diplomové práce Ing. Martinu Landovi, Ph.D. za odborné vedení, rychlé reakce na mé dotazy a ochotu hledat na ně odpovědi. Dále bych chtěl poděkovat Camilo Polimeris za napsání QGIS Processing Frameworku, jehož je tato práce součástí. V neposlední řadě bych chtěl poděkovat rodině a kamarádům za důvěru a podporu během studií.          %input file
\cleardoublepage
%%-----<<< ---------- >>>-----
%
%%-----<<< ABSTRAKT >>>-----
\section*{Abstrakt}\label{abstrakt}
\indent Diplomová práce si vytyčila za cíl vytvořit v prostředí Quantum GIS ( QGIS ) nástroj, který by umožňoval uživateli grafické propojování modulů z frameworku \textbf{QGIS Processing Framework}. V úvodní kapitole je představena knihovna Qt, resp. její verze PyQt pro jazyk Python, ve které byl celý Workflow Builder napsán. Dále je představeno prostředí QGIS a popsána práci s QGIS Processing Frameworkem jak z pozice uživatele GISu, tak z pozice vývojáře nových modulů. Dále je zmíněn SAGA Plugin, který byl napsán v rámci onoho frameworku.

V druhé kapitole diplomové práce je představena samotná aplikace Workflow Builder.

V poslední kapitole je zmínka o frameworku SEXTANTE, který se objevil v konci práce.

\textbf{Klíčová slova}

Quantum QGIS, workflow, open source, GIS, PyQt, SAGA, QGIS Processing Framework


\vfill

\section*{Abstract}
\indent

\textbf{Key words}

Quantum QGIS, workflow, open source, GIS, PyQt, SAGA, QGIS Processing Framework          %input file
\cleardoublepage
%%-----<<< -------- >>>-----
%
%%-----<<< ZADÁNÍ DIPLOMOVÉ PRÁCE >>>-----
%ZADANI - original
\cleardoublepage
%%-----<<< ---------------------- >>>-----
%
%%-----<<< TABLE OF CONTENTS >>>-----
\setcounter{secnumdepth}{4}             %number of section to 4
\setcounter{tocdepth}{4}                %number of section in table of contents greater then 3
\tableofcontents
\cleardoublepage
%%-----<<< ----------------- >>>-----
%
%%-----<<< TABLE OF CONTENTS PAGINATION >>>-----
\pagenumbering{arabic}                  %start arabic pagination from 1
%%-----<<< ---------------------------- >>>-----
%
%
%%-----<<< CHAPTERS >>>-----
\hyphenation{Automatica}                %no divide words
\pagestyle{fancy}

\chapter*{Úvod}
\addcontentsline{toc}{chapter}{Úvod}

V dnešní době se můžeme setkat s \index{GIS} geoinformačními ( GIS ) technologiemi na každém kroku. V různých oblastech krajinného inženýrství, při plánování výstavby silnic, v územním plánování, při řešení krizových situacích či plánování záchranných akcí. Uživatel si může vybrat z nepřeberného množství již existujících GIS nástrojů, řešení. A je pěkné, že svobodná řešení, nejen v oblasti geoinformačních technologií, drží krok s těmi proprietárními. Uživatel tedy nemusí sahat hluboko do kapsy. Co se týče nástrojů pro prohlížení, zpracování a analýzu \index{geodata} geodat, můžeme jmenovat například GRASS GIS, gvSIG, Quantum GIS či SAGA GIS. Tato práce si ale nekladla za cíl srovnat GIS nástroje, ale implementací nástroje do programu Quantum GIS, který by uživateli umožňoval vytvářet vlastní funkce spojováním již existujících funkcí.

Můžeme se také setkat s pojmy jako \begin{scriptsize}model builder\end{scriptsize} či \begin{scriptsize}chaining\end{scriptsize}, v té práci budu používat pojem workflow builder. Tento název byl převzat z projektu \index{VisTrails} VisTrails, který byl inspirací pro grafiku. Takzvané workflow buildery dávají uživateli možnost vytvářet si vlastní moduly za pomocí spojování výstupů a vstupů modulů již existujících. Uživatel tak nemusí spouštět každý modul zvlášť a starat se o výstupy, nová data, která se vytvoří jen dočasně a která uživatel v konečném výsledku nepotřebuje. Dále je pro uživatele je mnohem pohodlnější, pakliže může najít všechny funkce na jednom místě (tzv. toolbox), nežli při hledání procházet všechny možné pluginy.

V době psaní této diplomové práce existoval projekt \index{QGIS Processing Framework} QGIS Processing Framework studenta Camilo Polymeris z univerzity Universidad de Concepción. QGIS Processing Framework si kladl za cíl být frameworkem, který by sdružoval moduly z pluginů pro QGIS na jednom místě (tzv. toolbox). Odtud by byly jednotlivé moduly volány, pomocí workflow builderu spojovány, ukládány atp. V rámci tohoto projektu začala vznikat podpora pro použití modulů z jiného GIS nástroje - \index{SAGA GIS}System for Automated Geoscientific Analysis ( SAGA GIS ). V době psaní existovala podpora pro 170 modulů, ne všechny ale byly testovány a fungovaly správně. Ale i přesto se mohlo začít s práci na workflow builderu.

Aktuální verzi workflow builderu můžete najít zde:

\begin{center}
	\href{https://github.com/CzendaZdenda/qgis}{\texttt{https://github.com/CzendaZdenda/qgis}}
\end{center}              %input files
\chapter{Teorie}
V první části této kapitoly představím programovací jazyk Python, ve kterém byl napsán Workflow Builder. Jazyk Python je v dnešní době stále více oblíbený a můžeme ho najít snad všude, kam se podíváme. V druhé části této kapitole představím jeden ze svobodných systémů pro práci s geografickými daty - Quantum GIS a možnost rozšiřování funkcionality pomocí zásuvných modulů, tzv. pluginů. To bylo původně možné jen v jazyce C++. Již nějakou dobu je také možné psát pluginy v jazyce Python, což přineslo výhody v podobě jednoduché šiřitelnosti (není nutná kompilace) a snazšího vývoje pluginů. V další části se budu věnovat knihovně Qt, respektive její verze pro jazyk Python - PyQt4. Zde popíši nástroje, které jsem využil při psaní Workflow Builderu. Mezi tyto nástroje patří hlavně implementace architektury MVC v podobě model-view-delegate, Graphics View Framework pro vykreslování a správu grafických prvků, signály a sloty pro komunikaci mezi objekty knihovny Qt a mechanismus Drag and Drop. V předposlední části popíši projekt QGIS Processing Framework, co bylo jeho cílem, jeho koncem a také možnosti jeho rozšiřování.  V poslední části představím modul xml.dom jazyka Python pro práci s objekty ve formátu XML.

\newpage
\newpage
\section{Python}
\index{Python}
\nocite{py3:book}

Python je interpretovaný, procedurální a objektově orientovaný jazyk, ve kterém se rychle programuje. Existuje pythoní verze Qt - PyQt.
\newpage
\section{Quantum GIS}

\nocite{qgis:www}
\index{QGIS,Quantum GIS}

\begin{center}
	\includegraphics[scale=0.1]{pictures/qgis_logo}
\end{center}

Quantum GIS je nejen prohlížečka geografických dat dostupná pro MS Windows, GNU/Linux, Unix, BSD a Mac OS X. Quantum GIS podporuje díky knihovně \index{OGR}OGR většinu vektorových formátů dat jako například ESRI Shapefile, GRASS, MapInfo či GML a díky knihovně \index{GDAL} GDAL mnoho rastrových formátů jako TIFF, ArcInfo, GRASS raster, ERDAS a další. Přes Quantum GIS můžeme také přistupovat k datům uložených v geodatabázích PostGIS a SpatiaLite či k datům dostupných přes WMS a WFS služby. Quantum GIS je šířen pod licencí GNU Public Licence. \footnote{http://qgis.org/about-qgis/features.html}

Program je napsán v jazyce C++. Poslední stabilní verze nese označení 1.7.4. Quantum GIS je lehce rozšířitelný program pomocí pluginů, které mohou být psány na jazyce Python nebo C++. Quantum GIS má poměrně dobře zdokumentované API a nutno také podotknout, že komunita kolem Quantum GISu je aktivní a podpora skrze mailinglisty je na vysoké úrovni.

Systém začal vyvíjet v roce 2002 Gary Sherman. Mělo jít o nenáročnou prohlížečku geodat pro Linux s širokou podporou datových formátů. Dlouhou dobu byl Quantum GIS brán převážně jako grafická nadstavba pro jiný desktopový GIS - GRASS GIS. Přes GRASS Plugin QGIS zpřístupňuje řadu modulů GRASS GIS. V současnosti se na vývoji nejvíce podílí skupina vývojářu kolem organizace \footnotemark{\index{Faunalia} Faunalia}\footnotetext{http://www.faunalia.co.uk/quantumgis}.

Funkcionalitu Quantum GIS rozšiřuje množství pluginů. Jako základní pluginy bych označil \footnotemark{\index{fTools} \textbf{fTools}} \footnotetext{http://www.ftools.ca/}, který umožňuje pokročilé prostorové analýzy nad vektorvovými daty, \footnotemark{\index{GDAL, GdalTools} \textbf{GdalTools}} \footnotetext{http://www.faunalia.co.uk/gdaltools} pro práci s rastrovými daty a již zmíňený \footnotemark{\index{GRASS Plugin} \textbf{GRASS Plugin}} \footnotetext{http://grass.osgeo.org/wiki/GRASS\_and\_QGIS} plugin, který zpřístupňuje funkce GRASSu uživatelům Quantum GIS. 

\subsection{Správa pluginů}
Jak už bylo zmíněno Quantum GIS umožňuje uživatelům rozšiřovat funkce programu dle jejich potřeb v podobě zásuvných modulů. Díky dobře zdokumentovanému API může uživatel pohodlně psát pluginy v jazyce C++ nebo Python. Pluginy píší jak vývojáři Quantum GISu, tak i obyčejní uživatelé. Pluginy si můžete stáhnout z oficiálních či neoficiálních repositářů. Pro instalování pluginů napsaných v jazyce Python a správu repositářů slouží nástroj \textbf{QGIS Python Plugin Installer}, dostupný přes \textit{Plugins $\rightarrow$ Fetch Python Plugins...}.

\begin{figure}
	\centering
	\includegraphics[scale=0.5]{pictures/qgis_plugin/python_installer}
	\caption{QGIS Python Plugin Installer - správa repozitářů}
  	\label{pythonplugininstaller}
\end{figure}


\noindent Jak je vidět z [Obr. \ref{pythonplugininstaller}], takto nainstalované pluginy se stáhnou do adresáře: 

\begin{itemize}
	\item \textit{\$HOME$\setminus$.qgis$\setminus$python$\setminus$plugins} - v případě OS GNU/Linux
	\item \textit{C:$\setminus$Documents and Settings$\setminus$USER$\setminus$.qgis$\setminus$python$\setminus$plugins} - v případě OS Windows bývá cesta podobná této
\end{itemize}

Pakliže uživatel napíše plugin v jazyce Python, doporučuji ho uložit do tohoto adresáře. Je také sice možnost uložit plugin do adresáře \textit{\$QGIS\_INSTALL\_DIR $\setminus$share$\setminus$qgis$\setminus$python$\setminus$plugins}, ale při případné opětovné kompilaci by byly změny ztraceny.

\noindent Pluginy psané v C++ se po přeložení ukládají standardně v \textit{\$QGIS\_INSTALL\_DIR $\setminus$lib$\setminus$qgis$\setminus$plugins} případně uživatel může přidat nová úložiště pomocí \textit{Settings $\rightarrow$ Options} a v záložce \textit{Generals} zadat cestu [Obr. \ref{cpprepository}].

\begin{figure}
	\centering
	\includegraphics[scale=0.5]{pictures/qgis_plugin/options_cpp_path}
	\caption{\textit{Settings$\rightarrow$Options$\rightarrow$Generals} - přidání nové cesty k pluginům psaných na jazyce C++}
  	\label{cpprepository}
\end{figure}

Všechny nainstalované pluginy, ať psané v jazyce C++ či Python, může uživatel spravovat přes \textbf{QGIS Plugin Manager} - \textit{Plugins$\rightarrow$Manage Plugins...} [Obrázek \ref{plugin_manager}.

\begin{figure}
	\centering
	\includegraphics[scale=0.5]{pictures/qgis_plugin/plugin_manager}
	\caption{\textit{Plugins $\rightarrow$Manage Plugins...} - správa pluginů}
  	\label{plugin_manager}
\end{figure}

\subsection{Psaní vlastního pluginu}
Pluginy mohou být psány na jazyce C++ a Python. Již z charakteristiky daných jazyků vyplývá, že pro jednoduché, nenáročné či na začátku vývoje pluginy, se bude hodit spíše jazyk Python, který se nemusí kompilovat a píše se v něm rychleji než v jazyce C++. Pro rozsáhlejší projekty je lepší sáhnout po jazyce C++. Obecně jsou programy psané v kompilovaných jazycích mnohem rychlejší než programy psané v jazycích interpretovaných. 

\subsection{Python plugin}
\nocite{pyqgis:www}
\index{PyQGIS}
Při psaní pluginu v jazyce Python využíváme nástroje PyQGIS. Kromě dokumentace k Quantum GIS API také doporučuji \cite{pyqgis:www}. Dále můžeme využít nástroj Plugin Builder, což je v podstatě také plugin, který vygeneruje základní soubory, kód, který potom začneme upravovat podle tak, aby náš plugin dělal to, co chceme. \\

% dat to do prilohy
%\lstinputlisting[caption={\_\_init\_\_.py - inicializační soubor},label=plugin:init]{code/plugins/python/__init__.py}

\subsection{C++ plugin}
QGIS Processing Framework je plugin psaný v jazyce Python, proto se zde nebudu mnoho zmiňovat o pluginech psaných v jazyce C++. Více informací o tvorbě pluginů v C++ můžete najít v \footnotetext{http://download.osgeo.org/qgis/doc/manual/qgis-1.5.0\_coding-compilation\_guide\_en.pdf} \footnotemark{QGIS Coding and Compilation Guide}.
% 




\newpage
\section{Qt}

\subsection{Model/View Programming}
\subsection{Graphics View Interface}
\subsection{VisTrails, Orange}
\newpage
\index{QGIS Processing Framework}
\section{QGIS Processing Framework}
\nocite{pf:www}


\begin{center}
		\includegraphics[scale=0.30]{pictures/qgis_pf}
\end{center}

QGIS Processing Framework vznikl v rámci projektu GSoC
2011\footnote{Google Summer of Code. Projekt společnosti Google
na podporu studentu, více na \url{http://code.google.com/soc/}}. Student
Camilo Polymeris z univerzity Universidad de Concepción si kladl za
cíl napsat obecný framework, do kterého budou zapadat všechny moduly
všech pluginů QGISu a každý modul bude možné použít buď samostatně
nebo spojovat s jinými.

V době psaní této práce byla na světě první verze Processing
Frameworku a vše nasvědčovalo tomu, že práce na frameworku budou
pokračovat a nástroje v Processing Frameworku budou
přibývat. Existovala totiž pouze částečná podpora pro funkce SAGA
GIS a plugin zpřístupňující funkce \index{Orfeo Toolbox, OTB} Orfeo
Toolboxu (OTB). Orfeo Toolbox je svobodný software poskytující
nástroje pro zpracování snímku z \index{DPZ, dálkový průzkum Země}
dálkového průzkumu Země.

V době mého připojení k QGIS Processing Frameworku byl projekt na
začátku. Pro seznámení s projektem jsem přepsal Processing Manager
(toolbox) z QTreeWidget do MVC architektury.

\begin{figure}[h]
	\centering
	\includegraphics[scale=0.5]{pictures/pf/processing_manager_small}
	\caption{QGIS Processing Framework - Processing Manager}
  	\label{pf:pm}
\end{figure}

Processing Manager je část QGIS Processing Frameworku, která
zpřístupňuje všechny moduly dostupné skrze QGIS Processing Framework z
jednoho místa. Jedná se o panel se seznamem modulů, které jsou
rozděleny podle tagů do různých skupin (například 'raster',
'hydrology'). Každý modul obsahuje seznam tagů, které napovídají, k
čemu daný modul slouží. Uživatel může najít hledaný modul
prohledáváním samotného stromu, či využít vyhledávací okénko v horní
části panelu. Processing Manager prohledává tagy daného modulu a jeho
název. Modul obsahuje dále popis, ale protože se tagy generují z
tohoto popisu, není nutné popis procházet \figurename \ref{pf:pm}.

Modul je reprezentován třídou \textbf{Module} a jeho instance
třídou \textbf{ModuleInstace}. Z~třídy \textbf{Module} získáváme
informace o modulu. Pomocí metody $name$() získáme jméno modulu,
metoda $description$() vrací popis, metoda $tags$() a metoda
$instance$() vrací instanci třídy \textbf{ModuleInstace} daného
modulu. Zavoláním metody $parameters$() získáme seznam parametrů
daného modulu. Parametry jsou třídy \textbf{Parameter}.

U \textbf{ModuleInstace} můžeme pomocí metody $setValue$(Parameter,
hodnota) přiřazovat parametrům konkrétní hodnoty. Metodou
$value$(Paramter) získáme hodnotu parametru. Pomocí metody
$setState$() s parametrem "2"\,spouštíme daný modul. Metoda $module$()
vrací module (Module).

Parametry jsou třídy \textbf{Parameter}. Uchovávají v sobě informaci o
názvu a popisu parametru, zdali je parametr povinný, jakého je typu a
role (např. vstupní, výstupní) a jeho defaultní hodnotu. Přehled metod
třídy \textbf{Parameter} je zobrazen v [\tablename \ref{tab:metPar}].

\begin{table}[!]
	\centering
	\begin{tabular}{|c|c|}
	\hline
	{\bf metoda} & {\bf popis} \\
	\hline
	\hline
	name() & vrací jméno\\
	description() & vrací popis\\
	type() & vrací typ [viz Tabulka \ref{tab:pf_parametry} \\
	setRole(role) & nastaví roli\\
	role() & vrací roli\\
	setMandatory(bool) & nastaví zdali je parametr povinný\\
	isMandatory() & vrátí hodnotu, zdali je parametr povinný\\
	setDefaultValue() & nastaví defaultní hodnotu parametru\\
	defaultValue() & vrátí defaultní hodnotu parametru \\
	\hline
	\end{tabular}
	\caption{Metody třídy Parameter.}
	\label{tab:metPar}
\end{table}

Modulů z \textbf{QGIS Processing Framework} jsou přístupné buď přes \textbf{Processing Manager} nebo přes pythoní konzoli [\autoref{pf:konzole}]. Konzoli lze spustit například klávesovou zkratkou \textbf{Ctrl+Alt+ p}. 
\newpage

\begin{lstlisting}[label=pf:konzole,caption={Přístup k modulům přes konzoli.}] 
import processing

# seznam vsech registrovanych modulu
processing.framework.modules()
	
# vrati dany modulu
mod = processing.framework['nazev_modulu']
# vraci seznam parametru daneho modulu
mod.parameters()

# vytvori instanci modulu
instance = mod.instance()
# nastavi paramter
instance.setValue(parametr, hodnota)
# spusti modul
instance.setState(2)

# ziskani hodnoty parametru
instance.value(parametr2)
\end{lstlisting}

Spouští-li se modul přes \textbf{Processing Manager}, objeví se
dialogové okno pro nastavení parametrů modulu a následné spuštění
[viz \figurename \ref{pf:dialog}].

\begin{figure}[!]
	\centering
	\includegraphics[scale=0.5]{pictures/pf/processing_dialog}
	\caption{QGIS Processing Framework - okno pro nastavení a spuštění modulu}
  	\label{pf:dialog}
\end{figure}

%%%%%%%%%%%%%%%%%%%
%%% SAGA Plugin %%%
%%%%%%%%%%%%%%%%%%%
\newpage
\subsection{SAGA Plugin}
SAGA Plugin vznikl v rámci stejného projektu Camila Polymeris pro GSoC
2011. Měl zpřístupňovat funkce programu SAGA GIS pomocí jeho API
uživatelům Quantum GIS. Na stránkách
projektu \cite{pf:supportedModules} se deklaruje, že by mělo být
podporováno 170 modulů z celkových 425. Toto číslo vychází z
předpokladu, že moduly, u kterých jsou všechny vstupní i výstupní
parametry podporovány, pracují správně. Podporované parametry SAGA GIS
a jejich reprezentace v Processing
Frameworku \ref{tab:saga_parameters}. Parametry SAGA GIS, které nejsou
podporované Processing Frameworkem$:$ \textit{Table field, Data
Object, Grid list, Table, Node, Shape list, Parameters, Point Cloud,
TIN, Static table, Table list, Color, TIN list a Colors}. Dále nejsou
podporované interaktivní moduly. Bohužel ale nebyl plugin plně
dokončen a skutečný počet správně pracujících pluginů není roven
170. \\

\begin{table}[h]
	\centering
	\begin{tabular}{|c|c|}
		\hline
		\textbf{SAGA parametr} & \textbf{PF Parameter} \\
		\hline
		\hline
		Int & \multirow{3}{*}{NumericParameter}\\
		Double & \\
		Degree & \\
		\hline				
		Range & RangeParameter\\	
		\hline
		Bool & BooleanParameter\\		
		\hline
		String & \multirow{2}{*}{StringParameter}\\
		Text & \\
		\hline
		Chioce & ChoiceParameter\\
		\hline
		FilePath & PathParameter\\
		\hline
		Shapes & VectorLayerParameter\\
		\hline
		Grid & RasterLayerParameter\\		
		\hline	
	\end{tabular}
	\caption{parametry SAGA GIS podporované Processing Frameworkem}
	\label{tab:saga_parameters}
\end{table}


%%%%%%%%%%%%%%%%%%%%%%%%%%%%
%%% Psaní pluginu pro PF %%%
%%%%%%%%%%%%%%%%%%%%%%%%%%%%

\subsection{Psaní pluginu pro PF}
Plugin pro QGIS Processing Framework se v mnohém neliší od normálních
pluginů psaných pro QGIS. Inicializační soubor se prakticky vůbec
neliší. Třída reprezentující samotný plugin obsahuje navíc
metodu \textit{modules}(), která vrací seznam modulů třídy
processing. \textbf{Module} (dále jen Module), který poskytuje daný
plugin. Plugin samotný se musí registrovat v frameworku. To se provede
příkazem \begin{scriptsize}processing.framework.registerModule\-Provider(self)\end{scriptsize},
který se vloží do metody \textit{initGUI}().

Plugin tedy může obsahovat několik modulů, které vrací pomocí
metody \textit{modules}(). Každý modul se skládá ze sebe a ze svojí
"instance". Tedy z podtříd tříd Module a processing.ModuleInstance
(dále jen ModuleInstance). Module je pro daný modul základní třída,
která definuje typy parametry modulu, jeho název, popis a tagy. A
vrací jeho instanci v podobě ModuleInstance, která slouží ke spouštění
modulu s nastavenými parametry a spravuje, co se děje po provedení
modulu. To znamená, že pakliže chceme spustit modul, musíme nastavit
parametry a ty se nastavují v instanci, ne v~modulu samotném. Modul
poté spustíme metodou instance setStatus(2). Instance by mela
obsahovat kód, který vstupní parametry zpracuje a poté také nastaví
výstupní hodnoty.

Modul může mít několik vstupních a výstupních parametrů. V současné
době dovoluje framework uživateli použít
parametry \ref{tab:pf_parametry}.

\begin{table}	
	\centering
	\begin{tabular}{|c|c|c|}
		\hline
		\textbf{parametr} & \textbf{popis} & \textbf{grafická reprezentace}\\
		\hline
		\hline
		NumericParameter & číslo & QSpinBox\\
		\hline
		RangeParameter & dvojice číselných hodnot & pár QSpinBox\\	
		\hline
		BooleanParameter & boolean & QCheckBox\\		
		\hline
		\multirow{2}{*}{ChoiceParameter} & seznam možností & \multirow{2}{*}{QComboBox}\\
		& např. vrstev, metod & \\
		\hline
		StringParameter & textový řetězec & QLineEdit \\
		\hline
		PathParameter & cesta k souboru & QLineEdit + QPushButton \\
		\hline
		\multirow{2}{*}{VectorLayerParameter} & \multirow{2}{*}{QgsVectorLayer} & QComboBox s registrovanými \\
		& & vektorovými vrstvami\\
		\hline
		\multirow{2}{*}{RasterLayerParameter} & \multirow{2}{*}{QgsRasterLayer} & QComboBox s registrovanými \\
		& & rastrovými vrstvami\\		
		\hline	
	\end{tabular}
	\caption{parametry podporované Processing Frameworkem}
	\label{tab:pf_parametry}
\end{table}

Jako příklad uvedeme plugin Plugin, který bude mít na vstupu dva
parametry. Jeden parametr pro načtení cesty k rastrovému souboru a
druhý pro zadání jeho názvu, pod kterým se objeví v QGISu. Výstup bude
jeden - rastr třídy QgsRasterMapLayer. Příklad je pouze ilustrativní.

Inicializační soubor nebudu uvádět, protože se nijak neliší od toho,
když píšeme normální plugin pro QGIS. Soubor se samotným pluginem bude
obsahovat třídu Plugin, která reprezentuje náš plugin
[\autoref{pf:plugin}]. Dále třídu RasterToQgis(processing.\-Module)
[\autoref{pf:rasterToQgis}] a třídu
RasterToQgisInstance(processing.ModuleInstance)
[\autoref{pf:rasterToQgisInstance}].\\

\newpage
\begin{lstlisting}[caption={Třída Plugin pro QGIS Processing Framework}, label=pf:plugin, morekeywords={Plugin, __init__, modules, initGUI, unload}]
class Plugin:
    def __init__(self, iface):
        self.iface = iface
    def unload(self):
        pass
    def modules(self):
        return [self.rasterInputLayer]
    def initGui(self):
        self.rasterInputLayer = RasterToQgis(self.iface)
        processing.framework.registerModuleProvider(self)
\end{lstlisting}
\newpage
\begin{lstlisting}[caption={Třída RasterToQgis reprezentující modul pro QGIS Processing Framework}, label=pf:rasterToQgis, morekeywords={processing.Module,RasterToQgis, __init__, instance, RasterToQgisInstance, PathParameter, StringParameter, RasterLayerParameter}]
class RasterToQgis(processing.Module):
    def __init__(self, iface = None):
        self.iface = iface
        self.inParamPath = PathParameter("Path to input raster",
        	role=Parameter.Role.input)
        self.inParamName = StringParameter("Name of layerr",
        	role=Parameter.Role.input)
        self.outParam = RasterLayerParameter("Output raster",
        	role = Parameter.Role.output)
        self.outParam.setMandatory(False)
        processing.Module.__init__(self, "Input raster by path", 
            description = "Description",
            parameters = [self.inParamPath,self.inParamName,self.outParam], 
            tags = ["raster", "input"])

    def instance(self):
        return RasterToQgisInstance(self, self.inParamPath,
        							self.inParamName,  self.outParam)
\end{lstlisting}

V příkladu [\autoref{pf:rasterToQgis}] jsme u parametrů nastavili
pouze nejnutnější atributy jako název a roli. Role nám říká, zdali je
parametr povinný či volitelný. U parametrů můžeme nastavit také popis
či počáteční hodnotu při jejich tvorbě pomocí parametrů v
konstruktoru \textit{description} a \textit{defaultValue}. Nastavit
defaultní hodnotu můžeme také
metodou \textit{setDefaultValue}(value). \\

\begin{lstlisting}[caption={Třída RasterToQgisInstance reprezentující instanci modulu pro QGIS Processing Framework}, label=pf:rasterToQgisInstance, morekeywords={processing.ModuleInstance, RasterToQgisInstance, __init__, valueChangedSignal,onStateParameterChanged, QgsRasterLayer,setValue }]
class RasterToQgisInstance(processing.ModuleInstance):
    def __init__(self, module, inParamPath, inParamName,  outParam):
        self.inParamPath = inParamPath
        self.inParamName = inParamName
        self.outParam = outParam
        processing.ModuleInstance.__init__(self, module)
        QObject.connect(self,
            self.valueChangedSignal(self.stateParameter),
            self.onStateParameterChanged)
    def onStateParameterChanged(self, state):
        if state == StateParameter.State.running:            
            path = self[self.inParamPath]
            name = self[self.inParamName]
            raster = QgsRasterLayer(path, name)
            self.setValue(self.outParam, raster)
            self.setState(StateParameter.State.stopped)
\end{lstlisting}

Instance kontroluje stav (state) modulu. Je-li nastaven hodnotu 2
(StateParameter.State.running) vezme si vstupní parametry a na jejich
základě vytvoří novou rastrovou vrstvu \textbf{QgsRasterLayer}. A tu
nastaví do odpovídajícího výstupního parametru.

%%%%%%%%%%%%%%%%%%
%%% Conclusion %%%
%%%%%%%%%%%%%%%%%%

\subsection{Závěr}

Bylo by dobré vyřešit vstupní vrstvy aby například rastrová vrstva
zadaná jako PathParameter byla kompatibilní s parametrem
RasterLayerParameter. Dát vývojáři pluginu možnost, aby mohl uživatel
zadat vrstvu buď pomocí cesty nebo výběrem z již načtených vrstev. Dát
tedy uživateli obě možnosti.

Do této chvíle je napsán OTB Plugin pro zpracování družicových snímků
a rozepsán SAGA Plugin s podporou několika pluginů z SAGA GIS.

Camilo Polymeris měl v plánu pokračovat na projektu v rámci GSoC 2012,
ale po objevení frameworku SEXTANTE svoji žádost stáhl a zapojil se do
prací na SEXTANTE. QGIS Processing Framework se tedy zdá být mrtvým
projektem.

Dále bych chtěl upozornit, že během vývoje se změnila struktura
dat. Na začátku bylo jádro frameworku uloženo
v \begin{scriptsize}python/processing\end{scriptsize} a Processing
Manager, GUI a dialog pro spouštění modulů
v \begin{scriptsize}python/plugins/processingplugin\end{scriptsize}. V
současné době je vše uloženo
v \begin{scriptsize}python/processingmanager\end{scriptsize},
resp. jádro
v \begin{scriptsize}python/processingmanager/processing\end{scriptsize}. To
může na začátku psaní vlastního pluginu způsobit menší problém v
chybně zadané cestě k frameworku.

\newpage

\newpage
\section{.xml.dom.monidom}
\nocite{python:www}
\nocite{py3:book}

Nově vzniklé workflow se ukládají ve formátu \index{XML} XML. Výhoda tohoto formátu je, ze se s ním snadno pracuje a jde v podstatě pouze o textový dokument.

Python nabízí několik možností pro zpracování XML dokumentů.

XML nabízí jednoduché uložení hierarchicky strukturovaný dat. O prvcích XML dokumentu hovoříme jako elementech. Elementy jsou ohraničeny počátečními a koncovými znaky, tzv. tagy. XML dokument obsahuje vždy právě jeden kořenový element. Ten se může skládat z dalších a dalších elementů. Elementy mohou obsahovat atributy (dvojice jméno="hodnota").  Atributy se nesmějí v opakovat. Elementy kromě toho mohou obsahovat text. Text se uvádí mezi počáteční a koncový znak. \\

\noindent Příklad XML dokumentu:\\

\lstinputlisting[language=XML,caption={Příklad XML dokumentu},morekeywords={Graph, SubGraph, Module, tag}, label=xml:example]{code/plugins/xml/example_xml.xml}




\newpage
\chapter{Workflow Builder}

\section{Tvorba workflow}
\section{Uložení workflow}
\subsection{Popsání výstupního xml souboru}
XML nabízí jednoduché uložení hierarchicky strukturovaný dat. O prvcích XML dokumentu hovoříme jako elementech. Elementy jsou ohraničeny počátečními a koncovými značkami, tzv. tagy. XML dokument obsahuje vždy právě jeden kořenový element. Ten se může skládat z dalších a dalších elementů. V našem případě je kořenový element Graph. Ten se skládá z minimálně jednoho podgrafu (SubGraph), a ten poté minimálně z jednoho modulu (Module). Podgraf dále může obsahovat spojení mezi moduly (Connection). Modul kromě toho obsahuje elementy parametr (Port) a tag (tag) a popis. Graf také obsahuje tagy a popis. 

\begin{table}	
	\centering
	\begin{tabular}{|c|c|}
		\hline
		atribut & příklad \\
		\hline
		name & Addition two rasters \\
		tags & ['raster', 'hydrology'] \\	
		\hline	
	\end{tabular}
	\caption{atributy elementu Graph}
	\label{tab:graph}
\end{table}

\section{Načtení workflow do PF Manageru}

\newpage
\chapter{SEXTANTE}

\section{Srovnání QGIS Processing Framework v SEXTANTE}
%SEXTANTE podporuje daleko více modulů - GRASS, SAGA, OTB, GDAL Tools, fTools,... Ne všechny ale v době psaní této diplomové práce fungují správně. 



\section{Srovnání Workflow Builder v SEXTANTE Modeler}

\chapter*{Závěr}
Během práce na Workflow Builderu pro QGIS Processing Framework jsem se více seznámil s knihovnou Qt a jejím Graphics View Frameworkem.

Dále musím zmínit existenci druhého frameworku, který se objevil v konci psaní této práce. \index{SEXTANT}SEXTANT pro Quantum GIS. Daný framework má v podstatě podobné cíle a v současné době se zdá být on tou pravou cestou pro qgis, ikdyž ne všechny moduly jsou momentálně plně funkční. 

Někde jsem četl, že s architekturou MVC se dá seznámit za pár minut, ale naučit se ji správně využívat může trvat měsíce, i roky. Musím přiznat, že v mém případě to platí stoprocentně. Tudíž pakliže by projekt QGIS Processing Framework pokračoval, pokusil bych se stávající kód přepsat do podoby, která by splňovala všechny zásady a pravidla architektury MVC, tak jak nám umožňuje Qt.


\newpage
\pagenumbering{Roman}
%-----<<< SEZNAM UKAZEK KODU >>>-----
\lstlistoflistings  
%-----<<< REJSTRIK >>>-----
\addcontentsline{toc}{chapter}{Rejstřík}
\printindex
%-----<<< ---------- >>>-----

%-----<<< LITERATURA >>>-----
%\bibliographystyle{csplainnat}
\printbibliography
\addcontentsline{toc}{chapter}{Literatura}
%-----<<< ---------- >>>-----


\end{document}